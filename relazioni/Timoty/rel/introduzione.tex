\begin{abstract} 
    Come da consegna, è stato sviluppato un applicativo denominato \textbf{Kalk} che consiste 
    in una calcolatrice tra tipi di dato particolari. Il progetto è stato svolto in coppia da me,
    Timoty Granziero, e Andrea Nalesso. \par
    I tipi di calcolo scelti sono:
    \begin{itemize}
        \item \texttt{Matrix}, da cui derivano \texttt{SquareMatrix}
        e \texttt{SparseMatrix};
        \item \texttt{Network}, che usa la classe \texttt{User}.
    \end{itemize}
\end{abstract}

\section{Introduzione}

\subsection{Ambiente di sviluppo}
\paragraph*{Principale}
\begin{itemize}
    \item \textbf{Sistema Operativo}: Linux Mint 18.2 Sonya;
    \item Per \texttt{C++}:
    \begin{itemize}
        \item \textbf{Compilatore}: gcc 5.4.0;
        \item \textbf{Versione Qt Creator}: 4.0.2, basato su \textbf{Qt 5.7}.            
    \end{itemize}
    \item Per \texttt{Java}:
    \begin{itemize}
        \item \textbf{IDE}: Eclipse 
        \item \textbf{Versione}: Photon Milestone 5 (4.8.0M5)
    \end{itemize}
\end{itemize}

\subsection{Tempo di sviluppo} 
Quella che segue è una stima approssimativa del 
quantitativo effettivo di ore impiegato per lo sviluppo del progetto dal sottoscritto.
\begin{itemize}
    \item \textbf{Modello}:
    \begin{itemize}
        \item Analisi preliminare: 1 ora;
        \item Implementazione: 18 ore;
        \item Debugging e testing: 2 ore;
    \end{itemize}
    \item \textbf{GUI}:
    \begin{itemize}
        \item Analisi preliminare: 3 ore;
        \item Implementazione: 24 ore;
        \item Debugging e testing: 5 ore;
    \end{itemize}
\end{itemize}

\subsection{Suddivisione del lavoro progettuale}
Il progetto è stato svolto in coppia dal sottoscritto e Andrea Nalesso. La suddivisione del 
lavoro è stata la seguente:
\begin{itemize}
    \item \textbf{Timoty}:
    \begin{itemize}
        \item Gerarchia \texttt{Matrix} (\emph{model}) in \texttt{C++} e \texttt{Java};
        \item GUI (\emph{view} e \emph{controller}, utilizzando \texttt{Qt}) della gerarchia \texttt{Matrix};
    \end{itemize}
    \item \textbf{Andrea}:
    \begin{itemize}
        \item Contributo alla gerarchia \texttt{Matrix} (\emph{model}) in \texttt{C++};
        \item Gerarchia \texttt{Network} (\emph{model}) in \texttt{C++} e \texttt{Java};
        \item GUI (\emph{view} e \emph{controller}, utilizzando \texttt{Qt}) della gerarchia \texttt{Network}.
    \end{itemize}
\end{itemize}

\subsection{Compilazione ed esecuzione}
\paragraph{C\texttt{++}} Per compilare il progetto in \texttt{C++}, è stato fornito il file \texttt{kalk.pro}
all'interno della cartella \textbf{cpp}. Posizionarsi all'interno di quella cartella ed eseguire il comando
\texttt{qmake} per generare il Makefile, poi \texttt{make} per compilare i sorgenti. A questo punto, sar\`a presente
un eseguibile denominato \texttt{kalk}; è sufficiente eseguirlo con il comando \texttt{./kalk} (mentre si è posizionati
all'interno della cartella cpp) per lanciare l'applicativo.

\paragraph{Java} Per compilare il progetto in \texttt{Java}, compilare i sorgenti \texttt{java} dalla directory \textbf{java}
del progetto con il comando 
\begin{center}
    \texttt{javac kalkException/*.java dataTypes/*.java} \par
\end{center} 
Dopodichè eseguire lo script della classe \texttt{Use} invocando il comando (sempre dalla cartella java)
\begin{center}
    \texttt{java dataTypes.Use}
\end{center}
 