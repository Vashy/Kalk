\section{Descrizione e struttura del progetto}

\subsection{Design Pattern}
Il \emph{design pattern} scelto per lo sviluppo del progetto è il 
\emph{model-view-controller}, \textbf{MVC}. si è cercato di aderire
al più possibile a questo pattern architetturale.

\subsection{Modello}
Il \emph{modello} è la parte logica dell'applicativo. È stato implementato in
modo sia totalmente indipendente dalla \emph{view}. \par
Ogni modello di calcolo di Kalk deriva dalla classe \texttt{DataType}, una classe
astratta che non offre alcuna vera funzionalità (poichè si propone a un livello di astrazione
troppo alto, dato che da essa può derivare un qualunque tipo di classe. È stata usata solo per
raggruppare in modo coerente i tipi di dato usati nella calcolatrice), oltre a un \emph{distruttore %
virtuale puro}.
\subsubsection{Matrix}
Il primo tipo di calcolo è \texttt{Matrix}, una classe rappresenta le matrici matematiche.
Le celle sono memorizzate in un \texttt{QVector<double>}; sono presenti
i campi \texttt{row} e \texttt{col} che rappresentano le due dimensioni della matrice.
Essi sono accessibili in lettura attraverso le funzioni \texttt{rowCount()} e 
\texttt{colCount()}. \par

È possibile accedere alle celle con la funzione \texttt{double get(i,j)}, 
oppure attraverso l'operatore \texttt{[][]}: per implementarlo, sono presenti due classi interne
\texttt{Row} e \texttt{CRow} che si occupano di gestire la chiamata nitidificata dell'operatore 
\texttt{[]}. \texttt{CRow} è la versione costante di \texttt{Row}. L'operatore \texttt{[][]} comunque è meno
efficiente di \texttt{get(i,j)}, poichè passa attraverso l'istanziazione di un oggetto \texttt{Row/CRow}.
Allo stesso modo, è possibile accedere in scrittura alle celle con la funzione \mbox{\texttt{void set(i,j,k)}}, con 
\texttt{k} valore di tipo \texttt{double}.