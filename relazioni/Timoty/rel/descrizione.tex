\section{Descrizione e struttura del progetto}

\subsection{Design Pattern}
Il \emph{design pattern} scelto per lo sviluppo del progetto è il 
\emph{model-view-controller}, \textbf{MVC}. Si è cercato di aderire
il più possibile a questo pattern architetturale. \par
Il progetto è stato sviluppato seguendo l'ideologia dell'\emph{Open World Assumption}.
Per questa ragione, gran parte dei metodi sono dichiarati come \texttt{virtual}.\par
I file sono organizzati principalmente in 4 cartelle:
\begin{itemize}
    \item \textbf{model}\verb|/| contiene i file per il modello;
    \item \textbf{view}\verb|/| contiene i file relativi alla GUI implementata utilizzando 
    la libreria \texttt{Qt};
    \item \textbf{controller}\verb|/| contiene i file relativi ai controller;
    \item \textbf{exceptions}\verb|/| contiene i file relativi alla gerarchia delle eccezioni.
\end{itemize}


\subsection{Modello}
Il \emph{modello} è la parte logica dell'applicativo. È stato implementato in
modo che fosse totalmente indipendente dalla \emph{view}. \par
Ogni modello di calcolo di Kalk deriva dalla classe \texttt{DataType}, una classe
astratta che non offre alcuna vera funzionalità (poichè si propone a un livello di astrazione
troppo alto, dato che da essa può derivare un qualunque tipo di classe. È stata usata solo per
raggruppare in modo coerente i tipi di dato usati nella calcolatrice: ad esempio, \texttt{User},
che non è un tipo di dato utilizzato direttamente da Kalk, non deriva da \texttt{DataType}), 
oltre a offrire un \emph{distruttore virtuale puro}.
\subsubsection{Matrix}
Il primo tipo di calcolo è \texttt{Matrix}, una classe rappresenta le matrici matematiche.
Le celle sono memorizzate in un \texttt{QVector<double>}; sono presenti
i campi \texttt{row} e \texttt{col} che rappresentano le due dimensioni della matrice.
Essi sono accessibili in lettura attraverso le funzioni \texttt{rowCount()} e 
\texttt{colCount()}. \par

È possibile accedere alle celle con la funzione \texttt{double get(i,j)}, 
oppure attraverso l'operatore di subscripting \texttt{[][]}: per implementarlo, sono presenti due classi interne
\texttt{Row} e \texttt{CRow} che si occupano di gestire la chiamata nidificata dell'operatore 
\texttt{[]}. \texttt{CRow} è la versione costante di \texttt{Row}. L'operatore \texttt{[][]} comunque è meno
efficiente di \texttt{get(i,j)}, poichè passa attraverso l'istanziazione di un oggetto \texttt{Row/CRow}.
Allo stesso modo, è possibile accedere in scrittura alle celle con la funzione \mbox{\texttt{void set(i,j,k)}}, con 
\texttt{k} valore di tipo \texttt{double}.

%\paragraph{Operazioni di Kalk} 
Le operazioni che offre \texttt{Matrix} sono
la somma (\texttt{operator+}), la differenza (\texttt{operator-}), il prodotto scalare (\texttt{operator*})
definite come funzioni esterne; il prodotto non scalare (overload dell'\texttt{operator*}), la trasposta (\texttt{transposed()}), \texttt{swapRows} 
(scambia due righe passate come parametro, facendo \emph{side-effect}), \texttt{swapCols} (come la precedente 
ma sulle colonne) e \texttt{substituteRow} (somma la riga \emph{b} alla riga \emph{a} dopo aver moltiplicato 
\emph{b} per un \emph{double v}, facendo \emph{side-effect}) come funzioni interne;

\subsubsection{SquareMatrix} 
\texttt{SquareMatrix} è una classe derivata di \texttt{Matrix}. Non aggiunge nuovi \emph{field} all'oggetto,
è semplicemente un sottoinsieme delle matrici: matrici con le due dimensioni uguali.\par
Questo tipo di proprietà permette di aggiungere nuovi tipi di operazioni, che non sono possibili su una
normale matrice, ossia: il calcolo del determinante (\texttt{determinant()}); funzioni che restituiscono un 
booleano per determinare se la matrice è: triangolare superiore (\texttt{supTriangular()}), triangolare 
inferiore (\texttt{infTriangular()}), diagonale (\texttt{isDiagonal()}) e simmetrica (\texttt{isSymmetric()});
la funzione \texttt{getMinor()}, che restituisce una matrice con dimensione ridotta di uno rispetto alla 
\texttt{SquareMatrix} d'invocazione, in cui sono state rimosse la riga e la colonna passate come parametro. Sono infine presenti
\texttt{zeroMatrix()} e \texttt{identityMatrix}, due metodi statici che restituiscono una \texttt{SquareMatrix} di zeri o identità in 
base al valore intero passato come parametro.

\subsubsection{SparseMatrix}
\texttt{SparseMatrix} è un'altra classe derivata da \texttt{Matrix}. Una matrice sparsa è una normale matrice
che tendenzialmente ha molte celle con valore nullo (0). Aggiunge i campi \texttt{sparsity} e 
\texttt{dirtyBit}. Il primo viene inizialmente posto a $-1$, poichè il calcolo della sparsità avrebbe costo $O(n)$,
e per scelta progettuale questo calcolo è rimandato a quando la sparsità verrà esplicitamente richiesta. 
In caso la matrice venga modificata, il \texttt{dirtyBit} viene posto a \texttt{true} in modo da ricordare che
la sparsità va ricalcolata.\par
Le operazioni che offre sono: \texttt{getSparsity()} per ottenere il valore della sparsità (è un valore compreso tra 0 e 1 che è direttamente proporzionale al numero di celle nulle); \texttt{isDense()}
per sapere se la matrice è \emph{densa} (ossia sparsità $\leq 0.5$); \texttt{nonZeroRow()} restituisce i valori
(in un \texttt{QVector<double>}) $\neq 0$ nella riga passata come parametro, e l'analoga \texttt{nonZeroCol()},
che fa lo stesso ma con la colonna passata come parametro.

\paragraph*{Polimorfismo nella gerarchia Matrix}

\begin{itemize}
    \item Per la redifinizione degli operatori binari (somma, differenza, prodotto scalare) è stato scelto di seguire la 
    pratica comune del \texttt{C++}, senza utilizzare codice polimorfo, bensì dichiarando le funzioni esterne alla 
    classe, poichè in caso di polimorfismo si perderebbe la commutatività dell'operatore e bisognerebbe passare 
    attraverso puntatori e memoria allocata nello \emph{heap}, cosa che rischia di creare garbage.\par 
    Ad esempio (prendendo le versioni di \texttt{Matrix} e \texttt{SquareMatrix}), abbiamo, per \texttt{operator+}:
    \begin{center}
        \texttt{Matrix operator +(const Matrix}\verb|&|\texttt{, const Matrix}\verb|&|\texttt{)} \\
        \texttt{SquareMatrix operator +(const SquareMatrix}\verb|&|\texttt{, const SquareMatrix}\verb|&|\texttt{)}
    \end{center}
    
    \item Anche per l'operatore di assegnazione \texttt{operator=}, è stato scelto di non usare funzioni virtuali:
    questo perchè nel linguaggio non è necessario dichiarare virtuale tale operatore.

    \item \texttt{void Matrix::set()}, \texttt{double}\verb|&| \texttt{Matrix::getReference()} e altre funzioni che fanno \emph{side-effect}
    sulle celle della matrice: vengono \emph{overridate} in \texttt{SparseMatrix} perchè, nel caso in cui 
    venga modificato un oggetto che ha tipo dinamico \texttt{SparseMatrix},
    va tenuta traccia tramite il \texttt{dirtyBit} che è stata effettuata una modifica alla matrice, in modo
    da restituire un valore di \emph{sparsità} corretto in caso venga richiesto.

    \item \texttt{Matrix* Matrix::transposed()}: overridata in \texttt{SquareMatrix} e \texttt{SparseMatrix} 
    per restituire un puntatore con il tipo dinamico corretto. 
\end{itemize}
Per il resto, gran parte delle funzioni sono dichiarate come \texttt{virtual}, anche se non vengono esplicitamente overridate.

\subsubsection{Network}
\texttt{Network} è una classe che ha lo scopo di simulare una rete di \emph{utenti} interconnessi, in modo simile
a un \emph{social network} (ovviamente in forma molto semplificata).
L'utente è rappresentato tramite la classe \texttt{User}, composta da un campo \texttt{username}, che deve essere
univoco all'interno di una stessa rete. \par
Ogni rete è rappresentata da un oggetto \texttt{Network}. Per i puntatori usati in questa classe si è scelto
di usare gli smart pointer offerti da \texttt{Qt} con la classe \texttt{QSharedPointer}, per non aver problemi di gestione
della memoria. Una rete ha:
\begin{itemize}
    \item \texttt{QString name}: rappresenta il nome della rete;
    \item \texttt{QList<QSharedPointer<User>> userlist}: lista di utenti iscritti alla rete;
    \item \texttt{QList<QPair<QSharedPointer<User>, QSharedPointer<User>>> following}: lista di coppie
    \texttt{User*} - \texttt{User*}, in cui il primo utente ``segue'' il secondo.
\end{itemize}
La classe offre alcune funzioni di utilità, tra cui \texttt{addUser()}, \texttt{removeUser()}, \texttt{addFollower()}, \texttt{isFollowerOf()}, \texttt{size()} \dots

% Le funzioni di utilità che offre la classe sono:
% \begin{itemize}[noitemsep]
%     \item \texttt{size()} restituisce il numero di utenti nella rete;
%     \item \texttt{addUser()} aggiunge un utente alla rete (se non già presente, altrimenti lancia un'eccezione);
%     \item \texttt{isUserOfTheNetwork()} restituisce un booleano che è \texttt{true} se e solo se l'utente è presente
%     nella rete;
%     \item \texttt{removeUser()} rimuove un utente con un determinato \texttt{username};
%     \item \texttt{addFollower()} aggiunge una coppia (\texttt{QPair}) di puntatori a utenti a \texttt{followed} (il primo segue
%     il secondo);
%     \item \texttt{removeFollower()} rimuove una coppia da \texttt{following};
%     \item \texttt{isFollowerOf()} verifica se il primo \texttt{User} segue il secondo;
%     \item \texttt{getFollower()} restituisce un contenitore (\texttt{QSet}) con i puntatori a utenti seguiti da un certo \texttt{User};
%     \item \texttt{getFollowed()} restituisce un contenitore (\texttt{QSet}) con i puntatori a utenti che seguono un certo \texttt{User}.
% \end{itemize}

Infine sono presenti una serie di operazioni insiemistiche tra due \texttt{Network}: 
\begin{itemize}[noitemsep]
    \item \texttt{getUnion()} restituisce un \texttt{QSet} in cui sono presenti i puntatori a tutti gli utenti
    in comune e non tra le due reti;
    \item \texttt{getIntersection()} restituisce un \texttt{QSet} in cui sono presenti i puntatori a tutti gli 
    utenti in comune tra le due reti;
    \item \texttt{getRelativeComplement()} restituisce un \texttt{QSet} che è il complemento relativo tra gli utenti 
    delle due reti;
    \item \texttt{getSymmetricDifference()} restituisce un \texttt{QSet} che è la differenza simmetrica tra gli utenti
    delle due reti.
\end{itemize}

Tutti i metodi sono virtuali, ad eccezione di \texttt{size()}, poichè il numero di utenti sarà sempre quello di 
\texttt{userlist}, \texttt{getName()} e \texttt{setName()}, poichè il nome della rete resterà sempre la variabile
\texttt{name} nelle classi derivate.
 
\subsection{View}

Per implementare la GUI è stata utilizzata la libreria \texttt{Qt}. Non è stato utilizzato il tool \mbox{\emph{Qt Designer}} e il codice per la GUI è stato scritto
interamente a mano, poichè, data la scarsa complessità della GUI, sarebbe stato uno spreco di tempo dover andare a ritoccare a mano
il codice scritto in automatico dall'IDE.\par
Il widget iniziale è dato dalla classe 
\texttt{Widget} che deriva da \texttt{QWidget}. Da esso è possibile selezionare se usare \texttt{MatrixKalk} 
o \texttt{NetworkKalk}.

\subsubsection{Gerarchia MatrixKalk}
La prima classe con cui l'utente si trova ad interagire è \texttt{MatrixBuilder}, che deriva da \texttt{QWidget}.
Questo widget sarà l'istanza principale per \texttt{MatrixKalk}. Inizialmente permette di selezionare con quali 
matrici si vogliono effettuare operazioni di calcolo: ``Matrice'', ``Matrice Quadrata'' o ``Matrice Sparsa''.
In base a cosa l'utente sceglierà, verrà creata un'istanza della classe \texttt{MatrixKalk} per le matrici, 
\texttt{SquareMatrixKalk} (che deriva da \texttt{MatrixKalk}) per le matrici quadrate e \texttt{SparseMatrixKalk} 
(anch'essa deriva da \texttt{MatrixKalk}) per le matrici sparse, con relativi controller.\par
Per i campi di input numerici è stata implementata \texttt{KeypadInput}, derivata da \texttt{QLineEdit}, per poter controllare che non vengano inseriti
input non validi.

\paragraph{Polimorfismo in MatrixKalk} I metodi utilizzati per la costruzione (come ad esempio \texttt{buildDimensionsGroupBox()}) sono 
stati dichiarati non virtuali. Questo
perchè il linguaggio non permette di invocare in un costruttore un metodo virtuale (si otterrebbe un comportamento \emph{non} virtuale). \par
Metodi \emph{overridati} si hanno in \texttt{SquareMatrixKalk} (le funzioni, \emph{slot}, per la gestione della selezione delle dimensioni
e della moltiplicazione scalare, per gestire correttamente l'assenza del pulsante con la seconda dimensione). La totalità dei 
metodi per il calcolo delle operazioni (i relativi slot) sono definiti come \texttt{virtual}.

\subsubsection{NetworkKalk}
Per \texttt{NetworkKalk} tutte le operazioni vengono gestite attraverso un'unica grande finestra (classe \texttt{NetworkManager}), che permette di 
effettuare tutte le operazioni implementate:
è possibile creare e rimuovere utenti e reti, aggiungere o rimuovere un utente da una rete, aggiungere un utente creato a una rete e aggiungere e visualizzare i follower di un utente in una rete.\par
Infine, nel pannello in fondo, è possibile effettuare le operazioni insiemistiche tra due reti: unione, intersezione, complemento relativo e differenza simmetrica.\par
Per non sforare le ore a disposizione per il progetto, non è supportata la possibilità di 
salvare in memoria le informazioni su utenti e reti.

\subsection{Controller}

Il \emph{controller} si occupa di effettuare le operazioni di \texttt{Kalk} sui tipi di dato senza far interagire direttamente il modello con la vista.

\subsubsection{Gerarchia MatrixController}

Nella classe base \texttt{MatrixController} sono presenti due campi dato puntatore: \texttt{Matrix *matrix1, *matrix2}. Per accedere a questi campi sono 
presenti le funzioni \texttt{getMatrix()} in lettura, \texttt{buildMatrix()} e \texttt{setMatrix()} in scrittura, che accettano un parametro 
intero, \texttt{whichMatrix}, per decidere a quale puntatore accedere (per operazioni unarie si utilizza solo il primo). 
Sono poi presenti le funzioni per le operazioni tra matrici, tra cui \texttt{sum()}, \texttt{transposed()}, etc \dots \par
Da \texttt{MatrixController} derivano \texttt{SquareMatrixController} e \texttt{SparseMatrixController}, classi derivate del controller usate rispettivamente 
da \texttt{SquareMatrixKalk} e \texttt{SparseMatrixKalk} per gestire le operzioni assenti nella classe base del modello.

\paragraph{Polimorfismo} 
I metodi \texttt{getMatrix()}, \texttt{setMatrix()} e \texttt{buildMatrix()} sono dichiarati come virtuali e overridati nella classe 
\texttt{SparseMatrixController} per restituire un oggetto (allocato nello heap) del tipo dinamico adeguato.\par
I metodi di che si occupano di interfacciarsi con il modello, invece, non sono virtuali: questo perchè restituiscono un oggetto 
temporaneo e non un riferimento: nelle classi derivate andrebbe persa dell'informazione.

\subsubsection{NetworkController}
Il controller si occupa di effettuare tutte le operazioni sui dati (reti e utenti) sulla view, 
senza farla interagire direttamente con il modello.
\medskip

Praticamente tutte le funzioni di Network (view e controller) sono dichiarate virtuali.

\subsection{Eccezioni}

Le eccezioni usate in \texttt{Kalk} derivano tutte dalla classe astratta \texttt{KalkException}, la quale deriva a sua volta da \texttt{QException}.
\texttt{KalkException} aggiunge alla classe padre il campo \texttt{QString message} accessibile con il metodo \texttt{getMessage()}, e un 
costruttore a un parametro che permette di costruire il campo dati stringa in fase di costruzione. \par
Nella GUI, quando viene sollevata e catturata un'eccezione, viene invocata la funzione esterna \texttt{exceptionHandling(KalkException}\verb|&| \texttt{e)}, che si occupa di costruire un 
widget \texttt{QErrorMessage} che mostra la fonte del problema e l'errore riscontrato.

\section{Java}

In \texttt{Java} è stato implementato solo il modello di Kalk, e una classe \texttt{Use} che include il metodo \texttt{main()}, con una 
serie di esempi di utilizzo delle funzionalità del modello.\par
Il comportamento e l'implementazione sono pressochè identici: le differenze rispetto al \texttt{C++} sono limitate, in particolare 
sull'uso dei contenitori: per \texttt{Matrix} si è usato \texttt{ArrayList} invece di \texttt{QVector}, in \texttt{Network} sono stati 
usati \texttt{LinkedList}, \texttt{HashSet} e \texttt{AbstractMap.SimpleEntry} invece di \texttt{QList}, \texttt{QSet} e \texttt{QPair}, e sulle eccezioni, 
in cui le eccezioni usate sono raccolte nella cartella \textbf{kalkException}\verb|/| e derivano dalla classe \texttt{RuntimeException}.